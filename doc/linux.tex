\section{LINUX}

PLplot can be built on LINUX.  For basic info, see the section on Unix-type
systems earlier.  There are however some additional considerations which
bear mentioning.

First, LINUX does not currently support any video drivers.  Consequently the
only output options which are present in the makefile by default are those
for printers and the metafile.  This situation will hopefully be resolved
once the X-11 port to LINUX is completed.  If X is already available for
LINUX when you read this, you should be able to just add {\tt -DXWIN} to
the {\tt PLDEVICES} macro in the makefile.  In that event, be sure to fix
the link step to link against the X11 libraries as well.

Secondly, at the time of this writing (Spring '92), there are some problems
with some forms of binary file i/o on LINUX.  Specifically, using the
kernel 0.95c+ and the GCC 2.1 compiler, it is not possible to correctly
generate the fonts for PLplot.  This would appear to be totally crippling
since every PLplot program must be able to load the fonts.  The best
solution to this problem right now is to build PLplot on DOS, and copy the
{\tt *.fnt} files over to LINUX.  Hopefully this problem with the
interaction of GCC 2.1 and LINUX will be resolved soon, so this may not be
a problem when you read this.  You are welcome to attempt to build the
fonts and see what happens.  If you observe that {\tt plstnd4.fnt} has a
file size other than {\tt 6424} then you can be sure the fonts were not
built correctly.  In that case, build them on DOS and copy them over.
