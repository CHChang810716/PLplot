\section{OS/2}

At the time of this writing (Fall '91), the current release of OS/2 is version
1.3.  Consequently the current implementation of PLPLOT is 16-bit in nature,
and was developed using Microsoft 16 bit compilers.  Like all the rest of 
the industry, this author is anxiously awaiting the arrival of OS/2 2.0 with
its accompanying 32-bit programming model.  So, rest assured that there will
be a port of PLPLOT to OS/2 2.0 with all possible speed.  For the time being
PLPLOT does work with OS/2 1.3, and has been used with OS/2 1.2 as well.  
This author
has no experience with OS/2 versions prior to version 1.2, so no guarantees 
can be
made on that.

See the README file in {\tt sys$\backslash$os2} for additional 
information not covered herein.

\subsection{Compiler Support}

The OS/2 implementation of PLPLOT was developed using Microsoft C 6.0
and Microsoft Fortran 5.0 (these compilers were used for the DOS port
as well).  The majority of the code is of course independent of any
vendor specific compiler peculiarities, as evidenced by the wide range
of supported platforms.  However, in the case of OS/2 and DOS there is 
one notable exception.
Specifically, the way Microsoft chose to implement string argument 
passing in Fortran 5.0.  The details are rather involved, so will not
be given here.  Suffice it to say that it was necessary to write a 
special set of Fortran language interface stubs just for use with 
Microsoft Fortran 5.  If you wish to use some other compiler, you may
very well find that the Unix stubs work better.  It is genuinely hard
to imagine how anyone could write a compiler in such a way that would
make string handling any more difficult than Microsoft did in Fortran 5.0.

Further note that the supplied makefile is written with the syntax of
Microsoft NMAKE version 1.11.  Please do NOT attempt to use versions of
Microsoft NMAKE prior to 1.11, as both 1.0 and 1.1 have been known to
dramatically fail, and even to trash source files (through the misapplication,
of default rules).  It should not be
difficult to transform the provided makefile for use with other vendor's
make facilities.  If you do so, please send in the changes so they can
be provided to other users in future releases of this software.

\subsection{Installation}

Compiling the libraries is not too much work if you have the compilers
listed above.  Just get into the {\tt plplot$\backslash$tmp } directory, 
and issue the following commands:

\begin{verbatim}
copy ..\sys\os2\makefile .
nmake links
nmake
nmake fonts
nmake plrender
nmake cdemos fdemos
\end{verbatim}

The {\tt fonts } target bears special mention.  It compiles the programs which
create the fonts, runs them, and copies the resulting font files to 
{\tt $\backslash$lib }
on the current drive.  You should make sure you have such a directory before
you begin this process, or else change the makefile.

Once done building the software, you'll probably want to copy the files
{\tt plplot.lib } and {\tt plstub.lib } to someplace like {\tt $\backslash$lib } 
and  the {\tt plrender } utility to someplace in your path.

For programs to run, they need access to the fonts.  This is most easily
accomplished by setting the {\tt plfonts } environment variable.  If you
leave the fonts in the default place, the command would be:
\begin{verbatim}
set plfonts=d:\lib\
\end{verbatim}
where d is the drive where you'll be keeping the fonts.

\subsection{Linking}

Once the libraries are built, linking with them is not too difficult at all.

\begin{verbatim}
cl /AL main.c plplot.lib
fl /AL prog.for plplot.lib plstub.lib
\end{verbatim}

You will of course need to have the various compiler and linker recognized
environment variables correctly specified.  A sample file 
{\tt plplot$\backslash$sys$\backslash$os2$\backslash$plconfig.cmd } is 
provided to show how I do this.  Here 
is the important portion of the file:
\begin{verbatim} 

set	 path=d:\c6.0\binp;d:\c6.0\binb;c:\f5.0\bin;%syspath%
set	 path=%path%d:\util;c:\bndutil;d:\cmd

set	  lib=d:\c6.0\lib;c:\f5.0\lib
set   include=d:\c6.0\include;c:\f5.0\include
set helpfiles=d:\c6.0\help\*.hlp
set	   cl= /FPi87 /Lp
set	   fl= /FPi87 /link /NOE /NOD:llibfor7 /NOD:llibc7 llibc7p llibf7pc

set   plfonts=d:\lib\

\end{verbatim}
whwere {\tt syspath} is an environment variable containing all the directories
needed for the system to operate normally.
For descriptions of each of the various command options, see the compiler
reference materials.  Note that PLPLOT is compiled using the large memory 
model.  Note also that to use Microsoft Fortran with PLPLOT you 
will need to have the C compatible Fortran libraries built.

\subsection{Presentation Manager Support}

Providing PLPLOT graphics under OS/2 PM proved to be a demanding project.
The basic problem is that in order for an OS/2 program to interact with
PM, it must be event driven and must do all its work in response to messages
which are sent to it by the PM.  In other words, it is simply not possible
with the current version of OS/2 (1.3 at the time of this writing) to 
write a program which operates in the normal procedural paradigm, and which
makes calls to OS/2 services to draw things on the screen---that's just
not the way it works.  Rather a program which uses PM to draw windows and 
images must set up a message queue, and then go into an infinite loop
in which it obtains messages from the queue and processes them.  These
messages are sent to the program by PM, and tell it to do things like ``resize
thyself'',  ``redraw thyself'', ``redraw this subsection of window such
and such'', and so forth.  This style
of programming is basically the exact inverse of how normal scientific and
technical programs are constructed.  Furthermore, OS/2 PM applications 
cannot write to {\tt stdout}.  These restrictions conspire to make it
very difficult to generate visual graphics from a ``normal'' program when
run on OS/2.  (The same is true of MS Windows, but due to the notorious
instability of that environment, no effort has been made to solve the
problem in that context--so far).

In spite of the substantial difficulties involved, an output capability for
OS/2 PM has been developed which allows the user of PLPLOT to be totally
unconcerned with the inner workings of a native PM code.  That is, the 
user/programmer may write programs which call plplot functions, just as
s/he would on any other platform, and it is possible to view the output
in a window.  {\tt stdout} is also made available, as one would expect,
so that in fact no special programming whatsoever is required to make
a ``normal'' PLPLOT program work on OS/2.  

Due to the substantial effort which was required to provide a transparent
and high quality graphics output capability on OS/2, this PM specific
code (and documentation) is not being provided under the same distribution
terms as the rest of the PLPLOT package.  Rather a small fee of \$15 for
individual users is required.  Note, the free distribution of PLPLOT will 
run on OS/2---it just won't come with the PM driver.

To obtain the add-on package which will allow PLPLOT programs to generate
output for OS/2 Presentation Manager, individual users should mail \$15. to:
\begin{quote}
Physical Dynamics \\
P.O. Box 8556 \\
Austin, Texas~~~78713 \\
\end{quote}
Companies or institutions with multiple users may obtain site licenses for
the following rates:
\begin{quote}
First 100 users, \$15 each. \\
Next 100 users, \$10 each.  \\
Additional users, \$5 each. \\
\end{quote}
Texas residents add 8\% sales tax.  When you send in your money, be sure 
to specify your preferred delivery mechanism.  Options are: postal mailing
of the floppy of your choice, and internet e-mail of uuencoded ZOO, ZIP, 
etc.  Just
state your preference.  For those who request e-mail delivery, please include
a postal address in case there is any problem with the networks.
Companies requesting mulitple user site licenses should include a contact.

In brief, the capabilities of the OS/2 PM driver are:
\begin{itemize}

\item	Windowed output.  Like the X Windows driver.

\item	Image redraws when the window is exposed or resized.  This is the most
	important way in which the OS/2 PM driver is superior to the
        X Windows driver.

\item	Optional automatic frame advance.  The other major point of 
	functionality which the X driver doesn't have  (yet).
        
\item	Color selection compatible with the X driver.

\item	Minimizable/Maximizable, and preserves window contents  (just a
	special case of item 2 above).

\end{itemize}

If you choose not to obtain the OS/2 PM option  from Physical Dynamics, you
can certainly still use the free version of PLPLOT on OS/2.  The included 
metafile output option in PLPLOT allows you to generate a PLPLOT native
metafile from any OS/2 PLPLOT application.  You can then switch to a DOS box
and view the metafile using the {\tt plrender} utility.  You will need to
build a DOS version of {\tt plrender} once, and store it in a place accessible 
to your DOS path.  In fact, the PLPLOT metafile driver was originally invented 
specifically to allow viewing output in the DOS box from programs run in
OS/2 native mode, prior to the development of the OS/2 PM driver.

Be sure to check for README files in the 
{\tt plplot$\backslash$sys$\backslash$os2} directory for any additional
information not available at the time of preparation of this manual.
