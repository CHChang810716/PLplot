% -*-latexinfo-*-
% plplot.tex
% Geoffrey Furnish
% 9 May 1994
%
% This is a LaTeXinfo document detailing the use of PLplot.
%
% Process via:	% latex plplot
%
% $Id$
% $Log$
% Revision 1.3  1994/05/19 06:43:07  mjl
% Changed \documentstyle to include the plplot style file.  Not all
% sections included now; just the ones I'm working on (will need
% to be changed back).  Various fixes, updates, flashing lights...
%
% Revision 1.2  1994/05/09  22:09:06  furnish
% Improvements to the new doc.
%
% Revision 1.1  1994/05/09  15:06:25  furnish
% New files for building the PLplot manual via LaTeXinfo.
%
%%%%%%%%%%%%%%%%%%%%%%%%%%%%%%%%%%%%%%%%%%%%%%%%%%%%%%%%%%%%%%%%%%%%%

\documentstyle[12pt,fullpage,latexinfo,plplot]{book}
\pagestyle{headings}

\begin{document}

\c Declare which indices you want to make use of.
\newindex{cp}

\title{ The PLplot Plotting Library \\ 
        Programmer's Reference Manual \\
        Version 5.0}
\author{
        Maurice J. LeBrun\\
        Geoff Furnish\\
\\
        Institute for Fusion Studies\\
        University of Texas at Austin\\
        }

\date{\today}
\maketitle

\c The following commands start the copyright page for the printed manual.
\clearpage
\vspace{0pt plus 1filll}
Copyright \copyright\ 1994 Geoffrey Furnish, Maurice LeBrun.

PLplot may be used by anyone.

This manual may be copied by anyone.

Anyone may modify the program for personal use.

\c Permission is granted to copy and distribute modified versions of this
\c manual under the following five pages of conditions...

\c End the Copyleft page and don't use headings on this page.
\clearpage
\pagestyle{headings}

\pagenumbering{roman}
\tableofcontents

\c End the Table of Contents
\clearpage
\pagenumbering{arabic}

\setfilename{plplot.info}
\c Anything before the \setfilename will not appear in the Info file.

\begin{ifinfo}
Put another 5 page copyleft here for the Info file.
\end{ifinfo}

\c The Top node contains the master menu for the Info file.
\c This appears only in the Info file, not the printed manual.

\node Top,       Introduction, (dir),   (dir)

\begin{menu}
This document describes PLplot, the awesome free plotting library for
creating cool scientific plots.

* Introduction::	What PLplot is all about.
* Simple Use::		How to get started.
* Advanced Use::	How to go wild.
* X-windows Drivers::	Xwin, Tk, Tcl-DP, and all the rest.
* Extended WISH::	Integration with Tcl/Tk.
* Other Drivers::	All non-X drivers.
* API::			All the functions
* Bibliography::	What you don't want to know about refs.
* Concept Index::	All the stuff you need to know.
\end{menu}

\c %%%%%%%%%%%%%%%%%%%%%%%%%%%%%%%%%%%%%%%%%%%%%%%%%%%%%%%%%%%%%%%%%%%%%%

\part{Introduction}

\noinput{intro}

\part{Programming}

\c simple.tex
\c Geoffrey Furnish
\c 9 May 1994

\node Simple Use, Advanced Use, Introduction, Top
\chapter{Simple Use of PLplot}

This is the simple introduction to plotting with plplot.

\begin{menu}
* Simple Plot::		Plotting a Simple Graph
* Initializing::	PLplot
* Scales::		Defining Plot Scales and Axes
* Labelling::		The Graph
* Drawing::		The Graph
* Finishing::		Up...
* Error::		In case of...
\end{menu}

\c %%%%%%%%%%%%%%%%%%%%%%%%%%%%%%%%%%%%%%%%%%%%%%%%%%%%%%%%%%%%%%%%%%%%%%%%%

\c \chapter {Simple Use of PLPLOT} \label{simple}

\node Simple Plot, Initializing, Simple Use, Simple Use
\section{Plotting a Simple Graph}

We shall first consider plotting simple graphs showing the dependence
of one variable upon another. Such a graph may be composed of several
elements:
\begin{itemize}
   \item A box which defines the ranges of the variables, perhaps with axes
         and numeric labels along its edges, 
   \item A set of points or lines within the box showing the functional
         dependence, 
   \item A set of labels for the variables and a title for the graph.
\end{itemize}
In order to draw such a graph, it is necessary to call at least four of
the PLPLOT functions:
\begin{enumerate}
   \item {\tt plstar}, to specify the device you want to plot on, 
   \item {\tt plenv}, to define the range and scale of the graph, and
                      draw labels, axes, etc., 
   \item One or more calls to {\tt plline} or {\tt plpoin} to draw
         lines or points as needed. Other more complex routines include
         {\tt plbin} and {\tt plhist} to draw histograms, {\tt plerrx} and
         {\tt plerry} to draw error-bars, \label{draw}
   \item {\tt plend}, to close the plot.
\end{enumerate}
More than one graph can be drawn on a single set of axes by making
repeated calls to the routines listed in item \ref{draw} above.  PLPLOT 
only needs to be initialized once (via {\tt plstar} or a different startup
routine), unless it is desired to switch output devices between pages.

\c %%%%%%%%%%%%%%%%%%%%%%%%%%%%%%%%%%%%%%%%%%%%%%%%%%%%%%%%%%%%%%%%%%%%%%%%%%

\node Initializing, Scales, Simple Plot, Simple Use
\section{Initializing PLPLOT} 
\c \label{startingup}

Subroutine {\tt plstar} selects a graphics device or opens a disk file to
receive a plot for later display. If {\tt plstar} is called again during
a program, the previously opened file will be closed. When called, 
the user is prompted for a number representing the device on which
the plot is to appear. The syntax for {\tt plstar} is:

Xname{plstar(nx, ny);}
Xargu{{\tt nx, ny} (PLINT, input)}
{The number of plots to a page. The page is divided into
\c {\tt nx}~$\times$~{\tt ny} subpages, with {\tt nx} in the horizontal
direction, and {\tt ny} in the vertical direction.}

Subpages are useful for placing several graphs on a page, but all
subpages are constrained to be of the same size. For greater flexibility, 
read page \pageref{viewport} in Section
\ref{viewport} which discusses viewports.

An alternate startup routine {\tt plstart} behaves identically to {\tt
plstar} except that the device name is input by a keyword argument (keywords
are listed for each device in Appendix \ref{ap:dev}).  This style of call is
convenient if the user always outputs to the same device.  If the requested
device is not available, or if the input string is empty or begins with
``?'', the prompted startup of {\tt plstar} is used.  The syntax for {\tt
plstart} is:

Xname{plstart(devname, nx, ny);}
Xargu{{\tt devname} (char *, input)}
{The device name, in the form of a keyword for that device.}
Xargu{{\tt nx, ny} (PLINT, input)}
{The number of plots to a page, as in {\tt plstar}.}

The majority of calls to PLPLOT are made after initializing it by one of the
routines listed above, however, a few must be made {\em before\/} the
initialization in order to correctly set up the subsequent plots (the
description in the reference section identifies these setup calls).

\c %%%%%%%%%%%%%%%%%%%%%%%%%%%%%%%%%%%%%%%%%%%%%%%%%%%%%%%%%%%%%%%%%%%%%%%%%%

\node Scales, Labelling, Initializing, Simple Use
\section{Defining Plot Scales and Axes}

The function {\tt plenv} is used to define the scales and axes for simple
graphs. {\tt plenv} starts a new picture on the next subpage (or a new page
if necessary), and defines the ranges of the variables required. The
routine will also draw a box, axes, and numeric labels if requested.
The syntax for {\tt plenv} is:

Xname{plenv(xmin, xmax, ymin, ymax, just, axis);}
Xargu{{\tt xmin, xmax} (PLFLT, input)}
{The left and right limits for the horizontal axis.}
Xargu{{\tt ymin, ymax} (PLFLT, input)}
{The bottom and top limits for the vertical axis.}
Xargu{{\tt just} (PLINT, input)}
{This should be zero or one. If {\tt just} is one, the scales of the
x-axis and
y-axis will be the same (in units per millimeter); otherwise the axes are
scaled independently. This parameter is useful for ensuring that objects
such as circles have the correct aspect ratio in the final plot.}
Xargu{{\tt axis} (PLINT, input)}
{{\tt axis} controls whether a box, tick marks, labels, axes, and/or a
grid are drawn.
\begin{itemize}
     \item {\tt axis=-2}: No box or annotation.
     \item {\tt axis=-1}: Draw box only.
     \item {\tt axis= 0}: Draw box, labelled with coordinate values around edge.
     \item {\tt axis= 1}: In addition to box and labels, draw the two axes
                          X=0 and Y=0.
     \item {\tt axis= 2}: As for {\tt axis=1}, but also draw a grid at the major tick interval.
     \item {\tt axis=10}: Logarithmic X axis, linear Y axis.
     \item {\tt axis=11}: Logarithmic X axis, linear Y axis and draw line Y=0.
     \item {\tt axis=20}: Linear X axis, logarithmic Y axis.
     \item {\tt axis=21}: Linear X axis, logarithmic Y axis and draw line X=0.
     \item {\tt axis=30}: Logarithmic X and Y axes.
\end{itemize}
}

Note: Logarithmic axes only affect the appearance of the axes and their
labels, so it is up to the user to compute the logarithms prior to passing
them to {\tt plenv} and any of the other routines. Thus, if a graph has a
\c 3-cycle logarithmic axis from 1 to 1000, we need to set {\tt
\c xmin}$=\log_{10}1=0.0$, and {\tt xmax}$=\log_{10}1000=3.0$. 

For greater control over the size of the plots, axis labelling and tick
intervals, more complex graphs should make use of the functions
\c \rou{plvpor}, \rou{plvasp}, \rou{plvpas}, \rou{plwind}, \rou{plbox}, and
\c routines for manipulating axis labelling \rou{plgxax} through \rou{plszax},
described in Chapter \ref{reference}. 

\c %%%%%%%%%%%%%%%%%%%%%%%%%%%%%%%%%%%%%%%%%%%%%%%%%%%%%%%%%%%%%%%%%%%%%%%%%%

\node Labelling, Drawing, Scales, Simple Use
\section{Labelling the Graph}

The function {\tt pllab} may be called after {\tt plenv} to write labels
on the x and y axes, and at the top of the picture. All the
variables are character variables or constants. Trailing spaces
are removed and the label is centered in the appropriate field.
The syntax for {\tt pllab} is:

Xname{pllab(xlbl, ylbl, toplbl);}
Xargu{{\tt xlbl} (char *, input)}
{Pointer to string with label for the X-axis (bottom of graph).}
Xargu{{\tt ylbl} (char *, input)}
{Pointer to string with label for the Y-axis (left of graph).}
Xargu{{\tt toplbl} (char *, input)}
{Pointer to string with label for the plot (top of picture).}

\c More complex labels can be drawn using the function \rou{plmtex}.  See
Section \ref{graph-text} for information about the function {\tt plptex}
which writes labels within a graph, and section \ref{annotate} which
discusses floating point formats.

\c %%%%%%%%%%%%%%%%%%%%%%%%%%%%%%%%%%%%%%%%%%%%%%%%%%%%%%%%%%%%%%%%%%%%%%%%%%

\node Drawing, Finishing, Labelling, Simple Use
\section{Drawing the Graph}

PLPLOT can draw graphs consisting of points with optional error bars, 
line segments or histograms. Functions which perform each of these
actions may be called after setting up the plotting environment
using {\tt plenv}. All of the following functions draw within the box
defined by {\tt plenv}, and any lines crossing the boundary are clipped.
Functions are also provided for drawing surface and contour representations
of multi-dimensional functions. These are described in
Chapter \ref{advanced}.

\subsection {Drawing Points}
{\tt plpoin} and {\tt plsym} mark out {\tt n} points
{\tt (x[i], y[i])} with the
specified symbol. The routines differ only in the interpretation
of the symbol
codes. {\tt plpoin} uses an extended ASCII representation, with the
printable
ASCII codes mapping to the respective characters in the current font, and
the codes from 0--31 mapping to various useful symbols.
In {\tt plsym} however, 
the code is a Hershey font code number. Example programs are provided
which display each of the symbols available using these routines.

Xname{plpoin(n, x, y, code); {\rm and } plsym(n, x, y, code);}
Xargu{{\tt n} (PLINT, input)}
{the number of points to plot.}
Xargu{{\tt x, y} (PLFLT *, input)}
{pointers to arrays of the coordinates of the {\tt n} points.}
Xargu{{\tt code} (PLINT, input)}
{code number of symbol to draw.}

\subsection {Drawing Lines or Curves}

PLPLOT provides two functions for drawing line graphs.
All lines are drawn in
the currently selected color, style and width. See page \pageref{color} in
Section \ref{color}, page \pageref{style} in Section \ref{style} and
page \pageref{width} in Section \ref{width} for
information about changing these parameters.

{\tt plline} draws a line or curve. The curve consists of {\tt n-1} line
segments joining the {\tt n} points in the input arrays. For single line
segments, {\tt pljoin} is used to join two points.

Xname{plline(n, x, y);}
Xargu{{\tt n} (PLINT, input)}
{the number of points.}
Xargu{{\tt x, y} (PLFLT *, input)}
{pointers to arrays with coordinates of the {\tt n} points.}

Xname{pljoin(x1, y1, x2, y2);}
Xargu{{\tt x1, y1} (PLFLT, input)}
{coordinates of the first point.}
Xargu{{\tt x2, y2} (PLFLT, input)}
{coordinates of the second point.}

\subsection {Writing Text on a Graph} \label {graph-text}

{\tt plptex} allows text to be written within the limits set by
{\tt plenv}. The reference point of a text string may be located
anywhere along an imaginary horizontal line passing through the string
at half the height of a capital letter. The parameter {\tt just} specifies
where along this line the reference point is located. The string is then
rotated about the reference point through an angle specified by the
parameters {\tt dx} and {\tt dy}, so that the string becomes parallel
to a line joining {\tt (x, y)} to {\tt (x+dx, y+dy)}.

Xname{plptex(x, y, dx, dy, just, text);}
Xargu{{\tt x, y} (PLFLT, input)}
{coordinates of the reference point.}
Xargu{{\tt dx, dy} (PLFLT, input)}
{these specify the angle at which the text is to be printed. The text is
 written parallel to a line joining the points {\tt (x, y)} to
 {\tt (x+dx, y+dy)} on the graph.}
Xargu{{\tt just} (PLFLT, input)}
{determines justification of the string by specifying which point within
 the string is placed at the reference point {\tt (x, y)}. This parameter
 is a fraction of the distance along the string. Thus if {\tt just=0.0}, 
 the reference point is at the left-hand edge of the string.
 If {\tt just=0.5}, it is at the center and if {\tt just=1.0}, it is at
 the right-hand edge.}
Xargu{{\tt text} (char *, input)}
{pointer to the string of characters to be written.}

\subsection {Area Fills}

Area fills are done in
the currently selected color, line style, line width and pattern style.

{\tt plfill} fills a polygon. The polygon consists of {\tt n}
vertices which define the polygon.

Xname{plfill(n, x, y);}
Xargu{{\tt n} (PLINT, input)}
{the number of vertices.}
Xargu{{\tt x, y} (PLFLT *, input)}
{pointers to arrays with coordinates of the {\tt n} vertices.}


\subsection {More Complex Graphs (Histograms and Error Bars)}

\c Functions \rou{plbin} and \rou{plhist} are provided for
\c drawing histograms, and functions \rou{plerrx}
\c and \rou{plerry} draw error bars about specified
points. They are described in detail in Chapter \ref{reference}.

\c %%%%%%%%%%%%%%%%%%%%%%%%%%%%%%%%%%%%%%%%%%%%%%%%%%%%%%%%%%%%%%%%%%%%%%%%%%

\node Finishing, Error, Drawing, Simple Use
\section{Finishing Up}

\c Before the end of the program, {\em always\/} call \rou{plend} to close any
output plot files and to free up resources.  For devices that have separate
graphics and text modes, {\tt plend} resets the device to text mode. 

\c %%%%%%%%%%%%%%%%%%%%%%%%%%%%%%%%%%%%%%%%%%%%%%%%%%%%%%%%%%%%%%%%%%%%%%%%%%

\node Error, , Finishing, Simple Use
\section{In Case of Error}

If a fatal error is encountered during execution of a PLPLOT routine then
\c \rou{plexit} is called. This routine prints an error message, does resource
recovery, calls {\tt pl\_exit} (page~\pageref{plxexit}) and then exits. The
default {\tt pl\_exit} routine does nothing, but the user may wish to
supply his/her own version of {\tt pl\_exit} (for C programs only). 


\c advanced.tex
\c Geoffrey Furnish
\c 9 May 1994

\node Advanced Use, Extended WISH, Simple Use, Top
\chapter{Advanced Use of PLplot}

This explains how to do advanced stuff in plplot.


\noinput{drivers}

\noinput{wish}

\part{Reference}

\noinput{api}

\c %%%%%%%%%%%%%%%%%%%%%%%%%%%%%%%%%%%%%%%%%%%%%%%%%%%%%%%%%%%%%%%%%%%%%%%%%%

\part{Closure}

\bibliographystyle{plain}

\node Bibliography, Concept Index, API, Top

This is the bibliography.

\c \begin{thebibliography}{1}
\c 
\c \bibitem{birdsall}
\c C.~K. Birdsall and A.~B. Langdon.  {\it Plasma Physics via Computer
\c Simulation}.  McGraw Hill, New York. 1985.
\c 
\c \bibitem{hockney}
\c R.~W. Hockney and J.~W. Eastwood.  {\it Computer Simulation Using Particles}.
\c Adam Hilgar, Bristol and New York. 1988.
\c 
\c \bibitem{lebrun}
\c M.~J. LeBrun, M. G. Gray, G. Furnish, T. Tajima, and W. D. Nystrom  {\it
\c The Numerical ``Laboratory'' for Plasma Simulation: A Modern Approach
\c for the Condtruction and Maintenance of a Large Simulation Code}
\c Proceedings of the 14th International Conference on the Numerical
\c Simulation of Plasmas
\c 
\c \end{thebibliography}

\c  \node Concept Index, Top, First Chapter, Top
 \node Concept Index,, Bibliography, Top
\c  \unnumbered{Concept Index}

\twocolumn
\printindex{cp}

\end{document}
