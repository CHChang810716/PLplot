% -*-latexinfo-*-
% plplot.tex
% Geoffrey Furnish
% 9 May 1994
%
% This is a LaTeXinfo document detailing the use of PLplot.
%
% Process via:	% latex plplot
%
% $Id$
% $Log$
% Revision 1.1  1994/05/09 15:06:25  furnish
% New files for building the PLplot manual via LaTeXinfo.
%
%%%%%%%%%%%%%%%%%%%%%%%%%%%%%%%%%%%%%%%%%%%%%%%%%%%%%%%%%%%%%%%%%%%%%

\documentstyle[12pt,fullpage,latexinfo]{book}
\pagestyle{headings}

\begin{document}

\c Declare which indices you want to make use of.
\newindex{cp}

\title{ The PLPLOT Plotting Library \\ 
        Programmer's Reference Manual \\
        Version 4.0}
\author{
        Maurice J. LeBrun, 
        Geoff Furnish, and
        Tony Richardson$^1$\\
\\
        Institute for Fusion Studies\\
        University of Texas at Austin\\
        \\
        $^1$Dept of Electrical Engineering\\
        Duke University\\
        }

\date{\today}
\maketitle

\c The following commands start the copyright page for the printed manual.
\clearpage
\vspace{0pt plus 1filll}
Copyright \copyright\ 1994 Geoffrey Furnish, Maurice LeBrun, ???

PLplot may be used by anyone.

This manual may be copied by anyone.

Anyone may modify the program for personal use.

\c Permission is granted to copy and distribute modified versions of this
\c manual under the following five pages of conditions...

\c End the Copyleft page and don't use headings on this page.
\clearpage
\pagestyle{headings}

\pagenumbering{roman}
\tableofcontents

\c End the Table of Contents
\clearpage
\pagenumbering{arabic}

\setfilename{plplot.info}
\c Anything before the \setfilename will not appear in the Info file.

\begin{ifinfo}
Put another 5 page copyleft here for the Info file.
\end{ifinfo}

\c The Top node contains the master menu for the Info file.
\c This appears only in the Info file, not the printed manual.

\node Top,       Introduction, (dir),   (dir)

This document describes PLplot, the awesome free plotting library for
creating cool scientific plots.

\begin{menu}
* Introduction::	What PLplot is all about.

* Simple Use::		How to get started.

* Advanced Use::	How to go wild.

* Extended WISH::	Integration with Tcl/Tk

* API::			All the functions

* Bibliography::	What you don't want to know about refs.

* Concept Index::	All the stuff you need to know.
\end{menu}

\c %%%%%%%%%%%%%%%%%%%%%%%%%%%%%%%%%%%%%%%%%%%%%%%%%%%%%%%%%%%%%%%%%%%%%%

\part{Introduction}

\c intro.tex
\c Geoffrey Furnish
\c 9 May 1994

\node Introduction, Simple Use, Top, Top
\chapter{Introduction}
\cindex{PLPLOT Introduction}

This chapter introduces the incredibly cool PLplot scientific plotting
package.

\begin{menu}
* Library::		The PLplot Plotting Library
* Acquiring PLplot::	How to get it.
* Installing::		Installing and Using PLplot
* Organization::	Of this Manual
* Credits::		To whom it's due.
\end{menu}

\c \pagenumbering{arabic}
\node Library, Acquiring PLplot, Introduction, Introduction
\section{The PLPLOT Plotting Library}

PLPLOT is a library of C functions that are useful for making
scientific plots from a program written in C, C++, or Fortran.  The
PLPLOT library can be used to create standard x-y plots, semilog
plots, log-log plots, contour plots, 3D plots, mesh plots, bar charts
and pie charts.  Multiple graphs (of the same or different sizes) may
be placed on a single page with multiple lines in each graph.
Different line styles, widths and colors are supported.  A virtually
infinite number of distinct area fill patterns may be used.  There are
almost 1000 characters in the extended character set.  This includes
four different fonts, the Greek alphabet and a host of mathematical,
musical, and other symbols.  The fonts can be scaled to any desired
size.  A variety of output devices are supported and new devices can
be easily added by writing a small number of device dependent
routines.

Many of the underlying concepts used in the PLPLOT subroutine package
are based on ideas used in Tim Pearson's PGPLOT package originally
written in VAX-specific Fortran-77.  Sze Tan of the University of
Auckland originally developed PLPLOT on an IBM PC, and subsequently
transferred it to a number of other machines.  Additional features
were added to allow three-dimensional plotting and better access to
low-level routines.

The C version of PLPLOT was developed by Tony Richardson on a
Commodore Amiga.  In the process, several of the routines were
rewritten to improve efficiency and some new features added.  The
program structure was changed somewhat to make it easier to
incorporate new devices.

PLPLOT 4.0 is primarily the result of efforts by Maurice LeBrun and
Geoff Furnish of University of Texas at Austin to extend and improve
the previous work (PLPLOT 2.6b and 3.0, by Tony Richardson).  PLPLOT
4.0 is currently used as the main graphics engine for TPC (Toroidal
Particle Code), a large plasma simulation code developed at the IFS
\cite{lebrun89a}.  During this work we have found that PLPLOT compares
well with ``heavier'' packages (read: expensive, slow) and is an
excellent choice for scientists seeking an inexpensive (or free) but
high quality graphics package that runs on many different computing
platforms.

Some of the improvements in PLPLOT 4.0 include: the addition of
several new routines to enhance usage from Fortran and design of a
portable C to Fortran interface.  Additional support was added for
coordinate mappings in contour plots and some bugs fixed.  New
labelling options were added.  The font handling code was made more
flexible and portable.  A portable PLPLOT metafile driver and renderer
was developed, allowing one to create a generic graphics file and do
the actual rendering later (even on a different system).  The ability
to create family output files was added.  The internal code structure
was dramatically reworked, with elimination of global variables (for a
more robust package), the drivers rewritten to improve consistency,
and the ability to maintain multiple output streams added.  An XFig
driver was added.  Other contributions include Clair Nielsen's (LANL)
X-window driver (very nice for high-speed color graphics) and
tektronix file viewer.  At present, Maurice LeBrun and Geoff Furnish
are the active developers and maintainers of PLPLOT.

We have attempted to keep PLPLOT 4.0 backward compatible with previous
versions of PLPLOT.  However, some functions are now obsolete, and
many new ones have been added (e.g.  new contouring functions,
variable get/set routines, functions that affect label appearance).
Codes written in C that use PLPLOT must be recompiled including the
new header file ({\tt plplot.h}) before linking to the new PLPLOT
library.

PLPLOT is currently known to work on the following systems: SUNOS,
HP-UX, A/IX, DG/UX, UNICOS, Ultrix, VMS, Amiga/Exec, MS-DOS, OS/2, and
NeXT, with more expected.  The PLPLOT package is freely distributable,
but {\em not\/} in the public domain.  There have been various
copyrights placed on the software; see section \ref{sec:credits} for
the full list and criteria for distribution.

We welcome suggestions on how to improve this code, especially in the
form of user-contributed enhancements or bug fixes.  If PLPLOT is used
in any published papers, please include an acknowledgment or citation
of our work, which will help us to continue improving PLPLOT.  Please
direct all communication to:

\c \begin{tabbing}
\c 01234567\=
\c 	89012345678901234567890123456789\= \kill
\c %
\c 	\>Dr. Maurice LeBrun		\>Internet:\\
\c 	\>Institute for Fusion Studies	\>mjl@fusion.ph.utexas.edu\\
\c 	\>University of Texas\\
\c 	\>Austin, TX  78712\\
\c \\
\c 	\>Geoff Furnish			\>Internet:\\
\c 	\>Institute for Fusion Studies	\>furnish@fusion.ph.utexas.edu\\
\c 	\>University of Texas\\
\c 	\>Austin, TX  78712\\
\c \\
\c 	\>Tony Richardson		\>Internet:\\
\c 	\>184 Electrical Engineering	\>amr@egr.duke.edu\\
\c 	\>Duke University\\
\c 	\>Durham, NC 27706\\
\c \end{tabbing}

The original version of this manual was written by Sze Tan.

\node Acquiring PLplot, Installing, Library, Introduction
\section{Getting a copy of the PLPLOT package}

At present, the only mechanism we are providing for distribution of
the PLPLOT is by electronic transmission over the Internet.  We
encourage others to make it available to users without Internet
access.  PLPLOT may be obtained by {\tt ftp} from {\tt
hagar.ph.utexas.edu} (128.83.179.27).  Login as user {\tt anonymous},
set file transfer type to binary, and get the newest plplot archive in
the {\tt pub/} subdirectory.  We will provide PLPLOT in both {\tt zoo}
and {\tt tar} archives; get whichever one you prefer.

\node Installing, Organization, Acquiring PLplot, Introduction
\section{Installing and Using the PLPLOT Library}

The installation procedure is by necessity system specific;
installation notes for each system are provided in Appendix
\ref{ap:sys}.  The procedure requires that all of the routines be
compiled and they are then usually placed in a linkable library.

After the library has been created, you can write your main program to
make the desired PLPLOT calls.  Example programs in both C and Fortran
are included as a guide (if calling from C, you must include {\tt
plplot.h} into your program; see Appendix \ref{ap:lang} for more
details).  Plots generated from the example programs are shown at the
end of this work.

You will then need to compile your program and link it with the PLPLOT
library(s).  Again, please refer to the documentation specific to your
system for this.  Note that there may be more than one library
available to link with, such as single or double precision, with or
without X window libraries, IEEE floating point or Motorola FFP, etc.
Make sure you link to the correct one for your program.

\node Organization, Credits, Installing, Introduction
\section{Organization of this Manual}

The PLPLOT library has been designed so that it is easy to write
programs producing graphical output without having to set up large
numbers of parameters.  However, more precise control of the results
may be necessary, and these are accomodated by providing lower-level
routines which change the system defaults.  In Chapter \ref{simple},
the overall process of producing a graph using the high-level routines
is described.  Chapter\ref{advanced} discusses the underlying concepts
of the plotting process and introduces some of the more complex
routines.  Chapter \ref{reference} is the reference section of the
manual, containing an alphabetical list of the user-accessible PLPLOT
functions with detailed descriptions.

Because the PLPLOT kernel is written in C, standard C syntax is used
in the description of each PLPLOT function.  The C and Fortran
language interfaces are discussed in Appendix \ref{ap:lang}; look
there if you have difficulty interpreting the call syntax as described
in this manual.  The meaning of function (subroutine) arguments is
typically the same regardless of whether you are calling from C or
Fortran (but there are some exceptions to this).  The arguments for
each function are usually specified in terms of PLFLT and PLINT ---
these are the internal PLPLOT representations for integer and floating
point, and are typically a long and a float (or an INTEGER and a REAL,
for Fortran programmers).  See Appendix \ref{ap:lang} for more detail.

Also, you can use PLPLOT from C++ just as you would from C.  No
special classes are available at this time, just use it as any other
procedural type library.  Simply include {\tt plplot.h}, and invoke as
you would from C.

The output devices supported by PLPLOT are listed in Appendix
\ref{ap:dev}, along with description of the device driver--PLPLOT
interface, metafile output, family files, and vt100/tek4010 emulators.
In Appendix\ref{ap:sys} the usage and installation for each system
supported by PLPLOT is described (not guaranteed to be entirely
up-to-date; check the release notes to be sure).

\node Credits, , Organization, Introduction
\section{Credits}
\label{sec:credits}

PLPLOT 4.0 was created through the effort of many individuals and
funding agencies.  We would like to acknowledge the support (financial
and otherwise) of the following institutions:

\c \begin{description}
\c \item	The Institute for Fusion Studies, University of Texas at Austin
\c \item	The Scientific and Technology Agency of Japan
\c \item	Japan Atomic Energy Research Institute
\c \item	Duke University
\c \item	Universite de Nice
\c \item	National Energy Research Supercomputer Center
\c \item	Los Alamos National Labs
\c \end{description}

The authors disclaim all warranties with regard to this software,
including all implied warranties of merchantability and fitness, In no
event shall the authors be liable for any special, indirect or
consequential damages or any damages whatsoever resulting from loss of
use, data or profits, whether in an action of contract, negligence or
other tortious action, arising out of or in connection with the use or
performance of this software.

The PLPLOT source code, except header files and those files explicitly
granting permission, may not be used in a commercial software package
without consent of the authors.  You are allowed and encouraged to
include the PLPLOT object library and header files in a commercial
package provided that: (1) it is explicitly and prominently stated
that the PLPLOT library is freely available, and (2) the full
copyrights on the PLPLOT package be displayed somewhere in the
documentation for the package.

PLPLOT was first derived from the excellent PGPLOT graphics package by
T.  J. Pearson.  All parts of PLPLOT not explicitly marked by a
copyright are assumed to derive sufficiently from the original to be
covered by the PGPLOT copyright:
\begin{verbatim}
***********************************************************************
*                                                                     *
*  Copyright (c) 1983-1991 by                                         *
*  California Institute of Technology.                                *
*  All rights reserved.                                               *
*                                                                     *
*  For further information, contact:                                  *
*     Dr. T. J. Pearson                                               *
*     105-24 California Institute of Technology,                      *
*     Pasadena, California 91125, USA                                 *
*                                                                     *
***********************************************************************
\end{verbatim}

The code in PLPLOT not derived from PGPLOT is Copyright 1992 by
Maurice J.  LeBrun and Geoff Furnish of the University of Texas at
Austin and Tony Richardson of Duke University.  Unless otherwise
specified, code written by us as a part of this package may be freely
copied, modified and redistributed without fee provided that all
copyright notices are preserved intact on all copies and modified
copies.

The startup code for plrender.c is from {\tt xterm.c} of the X-windows
Release 5.0 distribution, and we reproduce its copyright here:
\begin{verbatim}
Copyright 1987, 1988 by Digital Equipment Corporation, Maynard, Massachusetts,
and the Massachusetts Institute of Technology, Cambridge, Massachusetts.

                        All Rights Reserved

Permission to use, copy, modify, and distribute this software and its 
documentation for any purpose and without fee is hereby granted, 
provided that the above copyright notice appear in all copies and that
both that copyright notice and this permission notice appear in 
supporting documentation, and that the names of Digital or MIT not be
used in advertising or publicity pertaining to distribution of the
software without specific, written prior permission.  

DIGITAL DISCLAIMS ALL WARRANTIES WITH REGARD TO THIS SOFTWARE, INCLUDING
ALL IMPLIED WARRANTIES OF MERCHANTABILITY AND FITNESS, IN NO EVENT SHALL
DIGITAL BE LIABLE FOR ANY SPECIAL, INDIRECT OR CONSEQUENTIAL DAMAGES OR
ANY DAMAGES WHATSOEVER RESULTING FROM LOSS OF USE, DATA OR PROFITS,
WHETHER IN AN ACTION OF CONTRACT, NEGLIGENCE OR OTHER TORTIOUS ACTION,
ARISING OUT OF OR IN CONNECTION WITH THE USE OR PERFORMANCE OF THIS
SOFTWARE.
\end{verbatim}

Thanks are also due to the many contributors to PLPLOT, including:

\c \begin{description}
\c \item Tony Richardson: Creator of PLPLOT 2.6b, 3.0
\c \item Sam Paolucci (postscript driver)
\c \item Clair Nielsen (X driver and tektronix file viewer)
\c \item Tom Rokicki (IFF driver and Amiga printer driver)
\c \end{description}

Finally, thanks to all those who submitted bug reports and other
suggestions.


\part{Programming}

\c simple.tex
\c Geoffrey Furnish
\c 9 May 1994

\node Simple Use, Advanced Use, Introduction, Top
\chapter{Simple Use of PLplot}

This is the simple introduction to plotting with plplot.

\begin{menu}
* Simple Plot::		Plotting a Simple Graph
* Initializing::	PLplot
* Scales::		Defining Plot Scales and Axes
* Labelling::		The Graph
* Drawing::		The Graph
* Finishing::		Up...
* Error::		In case of...
\end{menu}

\c %%%%%%%%%%%%%%%%%%%%%%%%%%%%%%%%%%%%%%%%%%%%%%%%%%%%%%%%%%%%%%%%%%%%%%%%%

\c \chapter {Simple Use of PLPLOT} \label{simple}

\node Simple Plot, Initializing, Simple Use, Simple Use
\section{Plotting a Simple Graph}

We shall first consider plotting simple graphs showing the dependence
of one variable upon another. Such a graph may be composed of several
elements:
\begin{itemize}
   \item A box which defines the ranges of the variables, perhaps with axes
         and numeric labels along its edges, 
   \item A set of points or lines within the box showing the functional
         dependence, 
   \item A set of labels for the variables and a title for the graph.
\end{itemize}
In order to draw such a graph, it is necessary to call at least four of
the PLPLOT functions:
\begin{enumerate}
   \item {\tt plstar}, to specify the device you want to plot on, 
   \item {\tt plenv}, to define the range and scale of the graph, and
                      draw labels, axes, etc., 
   \item One or more calls to {\tt plline} or {\tt plpoin} to draw
         lines or points as needed. Other more complex routines include
         {\tt plbin} and {\tt plhist} to draw histograms, {\tt plerrx} and
         {\tt plerry} to draw error-bars, \label{draw}
   \item {\tt plend}, to close the plot.
\end{enumerate}
More than one graph can be drawn on a single set of axes by making
repeated calls to the routines listed in item \ref{draw} above.  PLPLOT 
only needs to be initialized once (via {\tt plstar} or a different startup
routine), unless it is desired to switch output devices between pages.

\c %%%%%%%%%%%%%%%%%%%%%%%%%%%%%%%%%%%%%%%%%%%%%%%%%%%%%%%%%%%%%%%%%%%%%%%%%%

\node Initializing, Scales, Simple Plot, Simple Use
\section{Initializing PLPLOT} 
\c \label{startingup}

Subroutine {\tt plstar} selects a graphics device or opens a disk file to
receive a plot for later display. If {\tt plstar} is called again during
a program, the previously opened file will be closed. When called, 
the user is prompted for a number representing the device on which
the plot is to appear. The syntax for {\tt plstar} is:

Xname{plstar(nx, ny);}
Xargu{{\tt nx, ny} (PLINT, input)}
{The number of plots to a page. The page is divided into
\c {\tt nx}~$\times$~{\tt ny} subpages, with {\tt nx} in the horizontal
direction, and {\tt ny} in the vertical direction.}

Subpages are useful for placing several graphs on a page, but all
subpages are constrained to be of the same size. For greater flexibility, 
read page \pageref{viewport} in Section
\ref{viewport} which discusses viewports.

An alternate startup routine {\tt plstart} behaves identically to {\tt
plstar} except that the device name is input by a keyword argument (keywords
are listed for each device in Appendix \ref{ap:dev}).  This style of call is
convenient if the user always outputs to the same device.  If the requested
device is not available, or if the input string is empty or begins with
``?'', the prompted startup of {\tt plstar} is used.  The syntax for {\tt
plstart} is:

Xname{plstart(devname, nx, ny);}
Xargu{{\tt devname} (char *, input)}
{The device name, in the form of a keyword for that device.}
Xargu{{\tt nx, ny} (PLINT, input)}
{The number of plots to a page, as in {\tt plstar}.}

The majority of calls to PLPLOT are made after initializing it by one of the
routines listed above, however, a few must be made {\em before\/} the
initialization in order to correctly set up the subsequent plots (the
description in the reference section identifies these setup calls).

\c %%%%%%%%%%%%%%%%%%%%%%%%%%%%%%%%%%%%%%%%%%%%%%%%%%%%%%%%%%%%%%%%%%%%%%%%%%

\node Scales, Labelling, Initializing, Simple Use
\section{Defining Plot Scales and Axes}

The function {\tt plenv} is used to define the scales and axes for simple
graphs. {\tt plenv} starts a new picture on the next subpage (or a new page
if necessary), and defines the ranges of the variables required. The
routine will also draw a box, axes, and numeric labels if requested.
The syntax for {\tt plenv} is:

Xname{plenv(xmin, xmax, ymin, ymax, just, axis);}
Xargu{{\tt xmin, xmax} (PLFLT, input)}
{The left and right limits for the horizontal axis.}
Xargu{{\tt ymin, ymax} (PLFLT, input)}
{The bottom and top limits for the vertical axis.}
Xargu{{\tt just} (PLINT, input)}
{This should be zero or one. If {\tt just} is one, the scales of the
x-axis and
y-axis will be the same (in units per millimeter); otherwise the axes are
scaled independently. This parameter is useful for ensuring that objects
such as circles have the correct aspect ratio in the final plot.}
Xargu{{\tt axis} (PLINT, input)}
{{\tt axis} controls whether a box, tick marks, labels, axes, and/or a
grid are drawn.
\begin{itemize}
     \item {\tt axis=-2}: No box or annotation.
     \item {\tt axis=-1}: Draw box only.
     \item {\tt axis= 0}: Draw box, labelled with coordinate values around edge.
     \item {\tt axis= 1}: In addition to box and labels, draw the two axes
                          X=0 and Y=0.
     \item {\tt axis= 2}: As for {\tt axis=1}, but also draw a grid at the major tick interval.
     \item {\tt axis=10}: Logarithmic X axis, linear Y axis.
     \item {\tt axis=11}: Logarithmic X axis, linear Y axis and draw line Y=0.
     \item {\tt axis=20}: Linear X axis, logarithmic Y axis.
     \item {\tt axis=21}: Linear X axis, logarithmic Y axis and draw line X=0.
     \item {\tt axis=30}: Logarithmic X and Y axes.
\end{itemize}
}

Note: Logarithmic axes only affect the appearance of the axes and their
labels, so it is up to the user to compute the logarithms prior to passing
them to {\tt plenv} and any of the other routines. Thus, if a graph has a
\c 3-cycle logarithmic axis from 1 to 1000, we need to set {\tt
\c xmin}$=\log_{10}1=0.0$, and {\tt xmax}$=\log_{10}1000=3.0$. 

For greater control over the size of the plots, axis labelling and tick
intervals, more complex graphs should make use of the functions
\c \rou{plvpor}, \rou{plvasp}, \rou{plvpas}, \rou{plwind}, \rou{plbox}, and
\c routines for manipulating axis labelling \rou{plgxax} through \rou{plszax},
described in Chapter \ref{reference}. 

\c %%%%%%%%%%%%%%%%%%%%%%%%%%%%%%%%%%%%%%%%%%%%%%%%%%%%%%%%%%%%%%%%%%%%%%%%%%

\node Labelling, Drawing, Scales, Simple Use
\section{Labelling the Graph}

The function {\tt pllab} may be called after {\tt plenv} to write labels
on the x and y axes, and at the top of the picture. All the
variables are character variables or constants. Trailing spaces
are removed and the label is centered in the appropriate field.
The syntax for {\tt pllab} is:

Xname{pllab(xlbl, ylbl, toplbl);}
Xargu{{\tt xlbl} (char *, input)}
{Pointer to string with label for the X-axis (bottom of graph).}
Xargu{{\tt ylbl} (char *, input)}
{Pointer to string with label for the Y-axis (left of graph).}
Xargu{{\tt toplbl} (char *, input)}
{Pointer to string with label for the plot (top of picture).}

\c More complex labels can be drawn using the function \rou{plmtex}.  See
Section \ref{graph-text} for information about the function {\tt plptex}
which writes labels within a graph, and section \ref{annotate} which
discusses floating point formats.

\c %%%%%%%%%%%%%%%%%%%%%%%%%%%%%%%%%%%%%%%%%%%%%%%%%%%%%%%%%%%%%%%%%%%%%%%%%%

\node Drawing, Finishing, Labelling, Simple Use
\section{Drawing the Graph}

PLPLOT can draw graphs consisting of points with optional error bars, 
line segments or histograms. Functions which perform each of these
actions may be called after setting up the plotting environment
using {\tt plenv}. All of the following functions draw within the box
defined by {\tt plenv}, and any lines crossing the boundary are clipped.
Functions are also provided for drawing surface and contour representations
of multi-dimensional functions. These are described in
Chapter \ref{advanced}.

\subsection {Drawing Points}
{\tt plpoin} and {\tt plsym} mark out {\tt n} points
{\tt (x[i], y[i])} with the
specified symbol. The routines differ only in the interpretation
of the symbol
codes. {\tt plpoin} uses an extended ASCII representation, with the
printable
ASCII codes mapping to the respective characters in the current font, and
the codes from 0--31 mapping to various useful symbols.
In {\tt plsym} however, 
the code is a Hershey font code number. Example programs are provided
which display each of the symbols available using these routines.

Xname{plpoin(n, x, y, code); {\rm and } plsym(n, x, y, code);}
Xargu{{\tt n} (PLINT, input)}
{the number of points to plot.}
Xargu{{\tt x, y} (PLFLT *, input)}
{pointers to arrays of the coordinates of the {\tt n} points.}
Xargu{{\tt code} (PLINT, input)}
{code number of symbol to draw.}

\subsection {Drawing Lines or Curves}

PLPLOT provides two functions for drawing line graphs.
All lines are drawn in
the currently selected color, style and width. See page \pageref{color} in
Section \ref{color}, page \pageref{style} in Section \ref{style} and
page \pageref{width} in Section \ref{width} for
information about changing these parameters.

{\tt plline} draws a line or curve. The curve consists of {\tt n-1} line
segments joining the {\tt n} points in the input arrays. For single line
segments, {\tt pljoin} is used to join two points.

Xname{plline(n, x, y);}
Xargu{{\tt n} (PLINT, input)}
{the number of points.}
Xargu{{\tt x, y} (PLFLT *, input)}
{pointers to arrays with coordinates of the {\tt n} points.}

Xname{pljoin(x1, y1, x2, y2);}
Xargu{{\tt x1, y1} (PLFLT, input)}
{coordinates of the first point.}
Xargu{{\tt x2, y2} (PLFLT, input)}
{coordinates of the second point.}

\subsection {Writing Text on a Graph} \label {graph-text}

{\tt plptex} allows text to be written within the limits set by
{\tt plenv}. The reference point of a text string may be located
anywhere along an imaginary horizontal line passing through the string
at half the height of a capital letter. The parameter {\tt just} specifies
where along this line the reference point is located. The string is then
rotated about the reference point through an angle specified by the
parameters {\tt dx} and {\tt dy}, so that the string becomes parallel
to a line joining {\tt (x, y)} to {\tt (x+dx, y+dy)}.

Xname{plptex(x, y, dx, dy, just, text);}
Xargu{{\tt x, y} (PLFLT, input)}
{coordinates of the reference point.}
Xargu{{\tt dx, dy} (PLFLT, input)}
{these specify the angle at which the text is to be printed. The text is
 written parallel to a line joining the points {\tt (x, y)} to
 {\tt (x+dx, y+dy)} on the graph.}
Xargu{{\tt just} (PLFLT, input)}
{determines justification of the string by specifying which point within
 the string is placed at the reference point {\tt (x, y)}. This parameter
 is a fraction of the distance along the string. Thus if {\tt just=0.0}, 
 the reference point is at the left-hand edge of the string.
 If {\tt just=0.5}, it is at the center and if {\tt just=1.0}, it is at
 the right-hand edge.}
Xargu{{\tt text} (char *, input)}
{pointer to the string of characters to be written.}

\subsection {Area Fills}

Area fills are done in
the currently selected color, line style, line width and pattern style.

{\tt plfill} fills a polygon. The polygon consists of {\tt n}
vertices which define the polygon.

Xname{plfill(n, x, y);}
Xargu{{\tt n} (PLINT, input)}
{the number of vertices.}
Xargu{{\tt x, y} (PLFLT *, input)}
{pointers to arrays with coordinates of the {\tt n} vertices.}


\subsection {More Complex Graphs (Histograms and Error Bars)}

\c Functions \rou{plbin} and \rou{plhist} are provided for
\c drawing histograms, and functions \rou{plerrx}
\c and \rou{plerry} draw error bars about specified
points. They are described in detail in Chapter \ref{reference}.

\c %%%%%%%%%%%%%%%%%%%%%%%%%%%%%%%%%%%%%%%%%%%%%%%%%%%%%%%%%%%%%%%%%%%%%%%%%%

\node Finishing, Error, Drawing, Simple Use
\section{Finishing Up}

\c Before the end of the program, {\em always\/} call \rou{plend} to close any
output plot files and to free up resources.  For devices that have separate
graphics and text modes, {\tt plend} resets the device to text mode. 

\c %%%%%%%%%%%%%%%%%%%%%%%%%%%%%%%%%%%%%%%%%%%%%%%%%%%%%%%%%%%%%%%%%%%%%%%%%%

\node Error, , Finishing, Simple Use
\section{In Case of Error}

If a fatal error is encountered during execution of a PLPLOT routine then
\c \rou{plexit} is called. This routine prints an error message, does resource
recovery, calls {\tt pl\_exit} (page~\pageref{plxexit}) and then exits. The
default {\tt pl\_exit} routine does nothing, but the user may wish to
supply his/her own version of {\tt pl\_exit} (for C programs only). 


\c advanced.tex
\c Geoffrey Furnish
\c 9 May 1994

\node Advanced Use, Extended WISH, Simple Use, Top
\chapter{Advanced Use of PLplot}

This explains how to do advanced stuff in plplot.


\c wish.tex
\c Geoffrey Furnish
\c 9 May 1994

\node Extended WISH, API, Drivers, Top
\chapter{Building an Extended WISH}
\cindex{wish}
\cindex{Tcl Extensions}

Beginning with PLplot 5.0, a new and powerful paradigm for interaction
with PLplot is introduced.  This new paradigm consists of an
integration of PLplot with a powerful scripting language (Tcl), and
extensions to that language to support X Windows interface development
(Tk) and object oriented programming ([incr Tcl]).  Taken together,
these four software systems (Tcl/Tk/itcl/PLplot) comprise a powerful
environment for the rapid prototyping and development of
sophisticated, flexible, X Windows applications with access to the
PLplot API.  Yet that is only the beginning--Tcl was born to be
extended.  The true power of this paradigm is achieved when you add
your own, powerful, application specific extensions to the above
quartet, thus creating an environment for the development of wholly
new applications with only a few keystrokes of shell programming ...

\begin{menu}
* Tcl Intro::		The Tool Command Language
* Tk Intro::		User Interface Programming with Scripts
* Itcl Intro::		All the above with Object Orientation
* PLplot Extensions::	Using PLplot from Tcl.
* Custom Extensions::	Making your own WISH
\end{menu}

\node Tcl Intro, Tk Intro, Extended WISH, Extended WISH
\section{Introduction to Tcl}
\cindex{Tcl}
\cindex{What is Tcl}
\cindex{Programming with Tcl}

The Tool Command Language, or just Tcl (pronounced ``tickle'') is an
embedable script language which can be used to control a wide variety
of applications.  Designed by John Ousterhout of UC Berkeley, Tcl is
freely available under the standard Berkeley copyright.  Tcl and Tk
(described below) are extensively documented in a new book published by
Addison Wesley, entitled ``Tcl and the Tk toolkit'' by John
Ousterhout.  This book is a must have for those interested in
developing powerful extensible applications with high quality X
Windows user interfaces.  The discussion in this chapter cannot hope
to approach the level of introduction provided by that book.  Rather
we will concentrate on trying to convey some of the excitement, and
show the nuts and bolts of using Tcl and some extensions to provide a
powerful and flexible interface to the PLplot library within your
application.

\begin{menu}
* Motivation for Tcl::		Why Tcl exists
* Capabilities of Tcl::		What it can do for you
* Acquiring Tcl::		Where to get it, supporting tools, etc.
\end{menu}

\node Motivation for Tcl, Capabilities of Tcl, Tcl Intro, Tcl Intro
\subsection{Motivation for Tcl}

The central observation which led Ousterhout to create Tcl was the
realization that many applications require the use of some sort of a
special purpose, application specific, embedded ``macro language''.
Application programmers cobble these ``tiny languages'' into their
codes in order to provide flexibility and some modicum of high level
control.  But the end result is frequently a quirky and fragile
language.  And each application has a different ``tiny language''
associated with it.  The idea behind Tcl, then, was to create a single
``core language'' which could be easily embedded into a wide variety
of applications.  Further, it should be easily extensible so that
individual applications can easily provide application specific
capabilities available in the macro language itself, while still
providing a robust, uniform syntax across a variety of applications.
To say that Tcl satisfies these requirements would be a spectacular
understatement.

\node Capabilities of Tcl, Acquiring Tcl, Motivation for Tcl, Tcl Intro
\subsection{Capabilities of Tcl}

The mechanics of using Tcl are very straightforward.  Basically you
just have to include the file \file{tcl.h}, issue some API calls to
create a Tcl interpreter, and then evaluate a script file or perform
other operations supported by the Tcl API.  Then just link against 
\file{libtcl.a} and off you go.

Having done this, you have essentially created a shell.  That is, your
program can now execute shell scripts in the Tcl language.  Tcl
provides support for basic control flow, variable substitution file
i/o and subroutines.  In addition to the builtin Tcl commands, you can
define your own subroutines as Tcl procecdures which effectively
become new keywords.

But the real power of this approach is to add new commands to the
interpretter which are realized by compiled C code in your
application.  Tcl provides a straightforward API call which allows you
to register a function in your code to be called whenever the
interpretter comes across a specific keyword of your choosing in the
shell scripts it executes.

This facility allows you with tremendous ease, to endow your
application with a powerful, robust and full featured macro language,
trivially extend that macro language with new keywords which trigger
execution of compiled application specific commands, and thereby raise
the level of interaction with your code to one of essentially shell
programming via script editing.

\node Acquiring Tcl, , Capabilities of Tcl, Tcl Intro
\subsection{Acquiring Tcl}

There are several important sources of info and code for Tcl.
Definitely get the book mentioned above.  The Tcl and Tk toolkits are
distributed by anonymous ftp at \file{sprite.berkeley.edu:/tcl}.
There are several files in there corresponding to Tcl, Tk, and various
forms of documentation.  At the time of this writing, the current
versions of Tcl and Tk are 7.3 and 3.6 respectively.  Retrieve those
files, and install using the instructions provided therein.

The other major anonymou ftp site for Tcl is
\file{harbor.ecn.purdue.edu:/pub/tcl}.  Harbor contains a mirror of
\file{sprite} as well as innumerable extensions, Tcl/Tk packages,
tutorials, documentation, etc.  The level of excitement in the Tcl
community is extraordinarily high, and this is reflected by the great
plethora of available, high quality, packages and extensions available
for use with Tcl and Tk.  Explore--there is definitely something for
everyone.

Additionally there is a newsgroup, \file{comp.lang.tcl} which is well
read, and an excellent place for people to get oriented, find help,
etc.  Highly recommended.

In any event, in order to use the Tk and Tcl-DP drivers in PLplot, you
will need at a miniumum, \file{tcl7.3.tar.gz}, \file{tk3.6.tar.gz} and
\file{tcl-dp-3.1.tar.gz}.  Additionally, in order to use the extended
WISH paradigm (described below) you will need \file{itcl-1.5.tar.gz}.
Because of these dependencies, the PLplot ftp archive site provides
these files as well.  You can retrieve them when you get PLplot.

However, you will quite likely find Tcl/Tk to be very addictive, and
the great plethora of add-ons available at \file{harbor} wil
undoubtedly attract no small amount of your attention.  It has been
our experience that all of these extensions fit together very well.
You will  find that there are large sectors of the Tcl user community
which create so-called ``MegaWishes'' which combine many of the
available extensions into a single, heavily embellished, shell
interpretter.  The benefits of this approach will become apparent as
you gain experience with Tcl and Tk.

\node Tk Intro, Itcl Intro, Tcl Intro, Extended WISH
\section{Introduction to Tk}
\cindex{Tk}
\cindex{Programming Tk}

As mentioned above, Tcl is designed to be extensible.  The first and
most basic Tcl extension is Tk, an X11 toolkit.  Tk provides the same
basic facilities that you may be familiar with from other X11 toolkits
such as Athena and Motif, except that they are provided in the context
of the Tcl language.  There are C bindings too, but these are seldom
needed--the vast majority of useful Tk applications can be coded using
Tcl scripts.

If it has not become obvious already, it is worth noting at this point
that Tcl is one example of a family of languages known generally as
``Very High Level Languages'', or VHLL's.  Essentially a VHLL raises
the level of programming to a very high level, allowing very short
token streams to accomplish as much as would be required by many
scores of the more primitive actions available in a basic HLL.
consider, for example, the basic ``Hello World!'' application written
in Tcl/Tk.
\begin{verbatim}
#!/usr/local/bin/wish -f

button .hello -text "Hello World!"  -command "destroy ."
pack .hello
\end{verbatim}

That's it!  That's all there is to it.  If you have ever programmed X
using a traditional toolkit such as Athena or Motif, you can
appreciate how amazingly much more convenient this is.  If not, you
can either take our word for it that this is 20 times less code than
you would need to use a standard toolkit, or you can go write the same
program in one of the usual toolkits and see for yourself...

We cannot hope to provide a thorough introduction to Tk programming in
this section.  Instead, we will just say that immensely complex
applications can be constructed merely by programming in exactly the
way shown in the above script.  By writing more complex scripts, and
by utilizing the additional widggets provided by Tk, one can create
beautiful, extensive user interfaces.  Moreover, this can be done in a
tiny fraction of the time it takes to do the same work in a
conventional toolkit.  Literally minutes versus days.

Tk provides widgets for labels, buttons, radio buttons, frames with or
without borders, menubars, pull downs, toplevels, canvases, edit
boxes, scroll bars, etc.

A look at the interface provided by the PLplot Tk and Tcl-DP drivers
should help give you a better idea of what you can do with this
paradigm.  Also check out some of the contributed Tcl/Tk packages
available at harbor.  There are high quality Tk interfaces to a great
many familiar unix utilites ranging from mail to info, to SQL, to
news, etc.  The list is endless and growing fast...

\node Itcl Intro, PLplot Extensions, Tk Intro, Extended WISH
\section{Introduction to [incr Tcl]}
\cindex{[incr Tcl]}
\cindex{object oriented Tcl}

Another extremely powerful and popular extension to Tcl is [incr Tcl].
[incr Tcl] is to Tcl what C++ is to C.  The analogy is very extensive.
Itcl provides an object orientated extension to Tcl supporting
clustering of procedures and data into what is called an
\code{itcl_class}.  An \code{itcl_class} can have methods as well as
instance data.  And they support inheritance.  Essentially if you know
how C++ relates to C, and if you know Tcl, then you understand the
programming model provided by Itcl.

In particular, you can use Itcl to implement new widgets which are
composed of more basic Tk widgets.  A file selector is an example.
Using Tk, one can build up a very nice file selector comprised of more
basic Tk widgets such as entries, listboxes, scrollbars, etc.

But what if you need two file selectors?  You have to do it all again.
Or what if you need two different kinds of file selectors, you get to
do it again and add some incremental code.

This is exactly the sort of thing object orientation is intended to
assist.  Using Itcl you can create an \code{itcl_class FileSelector}
and then you can instantiate them freely as easily as:
\begin{verbatim}
    FileSelector .fs1
    .fs1 -dir . -find "*.cc"
\end{verbatim}
and so forth.

These high level widgets composed of smaller Tk widgets, are known as
``megawidgets''.  There is a developing subculture of the Tcl/Tk
community for designing and implementing megawidgets, and [incr Tcl]
is the most popular enabling technology.

In particular, it is the enabling technology which is employed for the
construction of the PLplot Tcl extensions, described below.

\node PLplot Extensions, Custom Extensions, Itcl Intro, Extended WISH
\section{PLplot Extensions to Tcl}
\cindex{PLplot Tcl extension}
\cindex{Tcl extension}

Following the paradigm described above, PLplot provides extensions to
Tcl as well, designed to allow the use of PLplot from Tcl/Tk programs.
Essentially the idea here is to allow PLplot programmers to achieve
two goals:
\begin{itemize}
\item To access PLplot facilties from their own extended WISH and/or
Tcl/Tk user interface scripts.

\item To have PLplot display its output in a window integrated
directly into the rest of their Tcl/Tk interface.
\end{itemize}

For instance, prior to PLplot 5.0, if a programmer wanted to use
PLplot in a Tcl/Tk application, the best he could manage was to call
the PLplot C api from compiled C code, and get the output via the Xwin
driver, which would display in it's own toplevel window.  In other
words, there was no integration, and the result was pretty sloppy.

With PLplot 5.0, there is now a supported Tcl interface to PLplot
functionality.  This is provided through a ``family'' of PLplot
megawidgets implemented in [incr Tcl].  Using this interface, a
programmer can get a PLplot window/widget into a Tk interface as
easily as:
\begin{example}
PLWin .plw
pack .plw
\end{example}
Actually, there's the update/init business--need to clear that up.

The \code{PLWin} class then mirrors much of the PLplot C API, so that
a user can generate plots in the PLplot widget entirely from Tcl.
This is demonstrated in the \file{tk02} demo,

\node Custom Extensions, , PLplot Extensions, Extended WISH
\section{Custom Extensions to Tcl}

By this point, you should have a pretty decent understanding of the
underlying philosophy of Tcl and Tk, and the whole concept of
extensions, of which [incr Tcl] and PLplot are examples.  These alone
are enough to allow the rapid prototyping and development of powerful,
flexible grpahical applications.  Normally the programmer simply
writes a shell script to be executed by the Tk windowing shell,
\code{wish}.  It is in vogue for each Tcl/Tk extension package to
build it's own ``extended WISH''.  Tcl-DP, for example builds
\code{dpwish} which users can use to evaluate scripts combining Tcl-DP
commands with standard Tcl/Tk.  There are many examples of this, and
indeed even PLplot's \code{plserver} program, described in an earlier
chapter, could just as easily have been called \code{plwish}.

In any event, as exciting and useful as these standalone, extended
windowing shells may be, they are ultimately only the beginning of
what you can do.  The real benefit of this approach is realized when
you make your own ``extended WISH'', comprised of Tcl, Tk, any of the
standard extensions you like, and finally embellished with a
smattering of application specific extensions designed to support your
own application domain.  In this section we give a detailed
introduction to the process of constructing your own WISH.  After
that, you're on your own...

\begin{menu}
* WISH construction::		How to write code for a private WISH
* WISH linking::		Turning your code into an executable
* WISH programming::		Using your new WISH
\end{menu}

\node WISH construction, WISH linking, Custom Extensions, Custom Extensions
\subsection{WISH Construction}
\cindex{making a wish}

The standard way to make your own WISH, as supported by the Tcl/Tk
system, is to take a boilerplate file, \file{tkAppInit.c}, edit to
reflect the Tcl/Tk extensions you will be requiring, add some commands
to the interpretter, and link it all together.

Here for example is the important part of the \file{tk02} demo,
extracted from the file \file{xtk02.c}, which is effectively the
extended WISH definition file for the \file{tk02} demo.  Comments and
other miscelany are omitted.
\begin{example}
#include "tk.h"
#include "itcl.h"

/* ... */

int   myplotCmd        (ClientData, Tcl_Interp *, int, char **);

int
Tcl_AppInit(interp)
    Tcl_Interp *interp;		/* Interpreter for application. */
\{
int   plFrameCmd        (ClientData, Tcl_Interp *, int, char **);

    Tk_Window main;

    main = Tk_MainWindow(interp);

    /*
     * Call the init procedures for included packages.  Each call should
     * look like this:
     *
     * if (Mod_Init(interp) == TCL_ERROR) \{
     *     return TCL_ERROR;
     * \}
     *
     * where "Mod" is the name of the module.
     */

    if (Tcl_Init(interp) == TCL_ERROR) \{
        return TCL_ERROR;
    \}
    if (Tk_Init(interp) == TCL_ERROR) \{
        return TCL_ERROR;
    \}
    if (Itcl_Init(interp) == TCL_ERROR) \{
        return TCL_ERROR;
    \}
    if (Pltk_Init(interp) == TCL_ERROR) \{
        return TCL_ERROR;
    \}

    /*
     * Call Tcl_CreateCommand for application-specific commands, if
     * they weren't already created by the init procedures called above.
     */

    Tcl_CreateCommand(interp, "myplot", myplotCmd,
                      (ClientData) main, (void (*)(ClientData)) NULL);


    /*
     * Specify a user-specific startup file to invoke if the
     * application is run interactively.  Typically the startup
     * file is "~/.apprc" where "app" is the name of the application.
     * If this line is deletedthen no user-specific startup file
     * will be run under any conditions.
     */

    tcl_RcFileName = "~/.wishrc";
    return TCL_OK;
\}

/* ... myPlotCmd, etc ... */
\end{example}

The calls to \code{Tcl_Init()} and \code{Tk_Init()} are in every WISH.
To make an extended WISH, you add calls to the initialization routines
for any extension packages you want to use, in this [incr Tcl]
(\code{Itcl_Init()}) and PLplot (\code{Pltk_Init()}).  Finally you add
keywords to the interpretter, associating them with functions in your
code using \code{Tcl_CreateCommand()} as shown.

In particular, PLplot has a number of [incr Tcl] classes in its Tcl
library.  If you want to be able to use those in your WISH, you need
to include the initialization of [incr Tcl].

\node WISH linking, WISH programming, WISH construction, Custom Extensions
\subsection{WISH Linking}

Having constructed your \code{Tcl_AppInit()} function, you now merely
need to link this file with your own private files to provide the code
for any functions you registered via \code{Tcl_CreateCommand()} (and
any they depend on), against the Tcl, Tk and extension libraries you
are using.
\begin{example}
cc -c tkAppInit.c
cc -c mycommands.c
cc -o my_wish tkAppInit.o mycommands.o 
           -lplplotftk -ltcl -ltk -litcl -lX11 -lm
\end{example}
Add any needed \code{-L} options as needed.

Voila!  You have made a wish.

\node WISH programming, , WISH linking, Custom Extensions
\subsection{WISH Programming}

Now you are ready to put the genie to work.  The basic plan here is to
write shell scripts which use your new application specific windowing
shell as their interpreter, to imlement X Windows user interfaces to
control and utilize the facilities made available in your extensions.

Effectively this just comes down to wrting Tcl/Tk code, embellished as
appropriate with calls to the extension commands you registered.
Additionally, since this wish includes the PLplot extensions, you can
instantiate any of the PLplot family of [incr Tcl] classes, and invoke
methods on those objects to effect the drawing of graphs.  Similarly,
you may have your extension commands (which are coded in C) call the
PLplot C programmers API to draw into the widget.  In this way you can
have the best of both worlds.  Use compiled C code when the
computational demands require the speed of compiled code, or use Tcl


\part{Reference}

\c api.tex
\c Geoffrey Furnish
\c 9 May 1994

\node API, Bibliography, Extended WISH, Top
\chapter{The PLplot Applications Programming Interface}

Documentation of each PLplot function.



\c %%%%%%%%%%%%%%%%%%%%%%%%%%%%%%%%%%%%%%%%%%%%%%%%%%%%%%%%%%%%%%%%%%%%%%%%%%

\part{Closure}

\bibliographystyle{plain}

\node Bibliography, Concept Index, API, Top

This is the bibliography.

\c \begin{thebibliography}{1}
\c 
\c \bibitem{birdsall}
\c C.~K. Birdsall and A.~B. Langdon.  {\it Plasma Physics via Computer
\c Simulation}.  McGraw Hill, New York. 1985.
\c 
\c \bibitem{hockney}
\c R.~W. Hockney and J.~W. Eastwood.  {\it Computer Simulation Using Particles}.
\c Adam Hilgar, Bristol and New York. 1988.
\c 
\c \bibitem{lebrun}
\c M.~J. LeBrun, M. G. Gray, G. Furnish, T. Tajima, and W. D. Nystrom  {\it
\c The Numerical ``Laboratory'' for Plasma Simulation: A Modern Approach
\c for the Condtruction and Maintenance of a Large Simulation Code}
\c Proceedings of the 14th International Conference on the Numerical
\c Simulation of Plasmas
\c 
\c \end{thebibliography}

\c  \node Concept Index, Top, First Chapter, Top
 \node Concept Index,, Bibliography, Top
\c  \unnumbered{Concept Index}

\twocolumn
\printindex{cp}

\end{document}
