\section{Unix-type systems}
\label{sec:unix}

%%%%%%%%%%%%%%%%%%%%%%%%%%%%%%%%%%%%%%%%%%%%%%%%%%%%%%%%%%%%%%%%%%%%%%%%%%%%%

\subsection{Linking}

Linking your program with the PLPLOT library on a Unix type
system can be done using something similar to (this is system specific):
\begin{verbatim}
cc  -o main main.c libplplotf.a -lm -lX11     or
f77 -o main main.f libplplotf.a -lm -lX11
\end{verbatim}
for the single precision library ({\tt -lplplotd} for double).  
You can also link using the {\tt -l} option to the linker,
i.e.:
\begin{verbatim}
cc  -o main main.c -lplplotf -lm -lX11     or
f77 -o main main.f -lplplotf -lm -lX11
\end{verbatim}
however in this case you must be very careful to get the order of libraries
on the command line right (libraries to be searched last should be placed at
the end of the command line).  In the latter case the library ({\tt
libplplotf.a}) must be in the search path used by the linker or specified by
the {\tt -L} option.

The utilities {\tt plrender} and {\tt pltek} should be in your search path.
Some good places to put these include {\tt /usr/local/bin/} or {\tt
/usr/local/plplot/}.

The PLPLOT font files must be where the library can find them.
The current directory is always searched for the fonts first, followed
by a system-dependent (and perhaps site-dependent) search path.
This can be modified from the makefile, but is structured so that it won't
often need to be modified.  Note that directory names must be defined with
the trailing slash, if non-null.  The default search order for Unix-type
systems is as follows:
\begin{verbatim}
        current directory
        $(HOME)/lib/
        $(PLFONTS)
        PLFONTDEV1      (/usr/local/lib/)
        PLFONTDEV2      (/usr/local/lib/plplot/
        PLFONTDEV3      (/usr/local/plplot/)
\end{verbatim}

This is will not be a concern to the user if the fonts are located correctly
during installation.

%%%%%%%%%%%%%%%%%%%%%%%%%%%%%%%%%%%%%%%%%%%%%%%%%%%%%%%%%%%%%%%%%%%%%%%%%%%%%

\subsection{Installation}

To achieve a degree of flexibility and system independence for the PLPLOT
makefile, it is written in the macro language m4.  System dependencies are
resolved conditionally via m4 macros.  Using this method, only one file
({\tt makefile.m4}) is required for all systems with a SystemV-compatible
{\tt make}.  This enables us to easily support many different systems.

Here's how to make everything conveniently on a unix-like system.  First,
if you run into problems, you should read {\tt sys/unix/makefile.m4}
carefully, as it has much additional information inside.  If your system is
not already supported, you must add system specific defines analogous to
the existing ones in {\tt makefile.m4}.  Then: 
\begin{verbatim}
% cd tmp
% cp ../sys/unix/makefile.m4 .
% m4 -D<sys> makefile.m4 >makefile 
\end{verbatim}
where $\langle$sys$\rangle$ is an abbreviation for your system name, e.g.
SUNOS, UNICOS, AIX, etc.  To get the double precision library for your
system, specify {\tt -DDOUBLE} on the {\tt m4} command line (note: Fortran
double precision may not be supported on your system using this method;
check the release notes to be sure).  At this point, you may wish to edit
{\tt makefile} to change whatever settings are required {\em for your site
only}.  For example, you may wish to edit the device list to include only
those devices which you actually have access to. 

Then, while in the {\tt tmp} directory:
%
\begin{verbatim}
% make links
\end{verbatim}
%
sets up soft links to all the files you will ever need.
%
\begin{verbatim}
% make [libs]
\end{verbatim}
%
will make the PLPLOT library(s) ({\tt libs} is the first target and
therefore optional), whereas
%
\begin{verbatim}
% make everything
\end{verbatim}
%
makes the main library, fonts, {\tt plrender}, and {\tt pltek}, for the
default precision (probably single on a workstation).  This is just an
abbreviation for typing {\tt \% make libs fonts plrender pltek}.
To make one of the example programs,
%
\begin{verbatim}
% make x01c
\end{verbatim}
%
for the first C demo, or
%
\begin{verbatim}
% make x01f
\end{verbatim}
%
for the first Fortran demo (other example programs are similar).  To make
all the C program or Fortran program demos, specify {\tt cdemos} or
{\tt fdemos} as the target.

Finally, you must move the PLPLOT library, fonts, and utilities to 
a more permanent location, for example:
\begin{verbatim}
% mv ../lib/*.lib /usr/local/plplot
% mv *.fnt *.idx *.lkp /usr/local/plplot
% mv plrender pltek /usr/local/plplot
\end{verbatim}
if you are installing into system directories, or
\begin{verbatim}
% mv ../lib/*.lib ~/lib
% mv *.fnt *.idx *.lkp ~/lib
% mv plrender pltek ~/bin
\end{verbatim}
if you are installing in your directory tree only (note under SUNOS you must
run {\tt ranlib} on libraries after moving them).  You should then test the
library again using one of the example programs to ensure that everything
works perfectly.  After you are finished, you may delete everything in the
tmp directory.

Note that starting with PLPLOT version 4.0, the {\tt *.idx} and the {\tt
*.lkp} files are no longer generated.  The info they contained has been
stuffed into the {\tt *.fnt} files.  This means when you move the font
files to their permanent locale, you only need to do
\begin{verbatim}
mv *.fnt wherever
\end{verbatim}

If you have any difficutly with installing the PLPLOT files into their
permanent places, your local unix system wizard may be able to help you.
